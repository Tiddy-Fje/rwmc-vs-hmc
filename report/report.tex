\documentclass[a4paper, 12pt,oneside]{article} 
%\documentclass[a4paper, 12pt,oneside,draft]{article} 
\usepackage{preamble}
%--------------------- ACTUAL FILE ---------------------- %
\begin{document} 
%%%
	%\setcounter{page}{1}
	\begin{center}
	    \Large
	    \textbf{Hamiltonian Monte Carlo} 
	    \vspace{0.4cm}
	    \large

		Course : Stochastic Simulation \\
	    Students : Aude Maier \& Tara Fjellman \\
	    \small{Fall 2024}
	\end{center}

	\section{Introduction}
	\section{Section ...}
        %\begin{figure}[htb]
        %    \centering
        %        \vspace{0em}
        %        \includegraphics[width=0.6\textwidth]{figures/fig1.pdf}
        %        \caption{Hello there.}
        %        \label{fig:label}
        %\end{figure}
	\section{(c)}
        \subsection{Gibbs distribution invariance}
		Under the assumption that there is no numerical error, we want to prove that the Gibbs measure is invariant for the chain generated by the Hamiltonian dynamics.

		This is equivalent to saying that the distribution $\pi$ is the same before and after an evolution of $t$ seconds from the hamiltonian dynamics. This can be expressed in terms of an equation as 
		\begin{gather}
			\pi(D)=\int_{\Omega} K((q,p),D) G(q,p) \ dq dp=\pi(T_t[D]),
		\end{gather}
		with $\Omega$ the phase space, $K:\Omega \times D\to [0,1]$ the transition kernel of the chain, $G(q,p)$ the Gibbs distribution, and $T_t[\cdot]$ the hamiltonian time evolution operator defined by $T[(q_s,p_s)]=(q_{s+t},p_{s+t})$.

		We follow the hind and consult theorem 2.3 of [cite] 
        \subsection{What happens for discrete evolution ?}
	\section{Section ...}
	\section{Conclusion}
	\section*{Aknowledgements}
	\section*{References}
	%\appendix
	%	\section{Runtime Estimation}\label{appendix:runtime_estimation}
%%%
\end{document} 

